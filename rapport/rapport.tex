\documentclass{article}

\usepackage[utf8]{inputenc}
\usepackage[french]{babel}

\usepackage{listings}
\usepackage{hyperref}
\usepackage{graphicx}
\usepackage{placeins}



\title{Rapport Traitement des données in-situ \break Équipe Firewall}
\author{Valentin Jonquière, Mathilde Chollon, Calliste Boudoux d'Hautefeuille}

\begin{document}

\maketitle
\pagebreak

\tableofcontents

\pagebreak

\section{Organisation du travail}
Nous travaillons sur le projet grâce à GitHub. Lors de chaque TD, nous créons des issues, que nous nous répartissons dans le Kanban.
Chaque membre du groupe possède un bout du TD à réaliser et aide les autres s'il finit en avance. Nous avons des branches pour chaque
 fonctionnalité et nous utilisons les merge requests afin de vérifier le travail des autres membres. Cela nous permet d'avoir du code
 fonctionnel dans la branche principale et de vraiment savoir ce que modifie chaque personne.


\section{Workflow ad-hoc}

\subsection{Snaphots}

\subsection{Sismos}

\section{Workflow in-situ}

\section{Comparaison workflows ad-hoc et in-situ}

\end{document}