\documentclass{article}

\usepackage[utf8]{inputenc}
\usepackage[french]{babel}

\usepackage{listings}
\usepackage{hyperref}
\usepackage{graphicx}
\usepackage{placeins}



\title{Rapport Traitement des données in-situ \break Équipe Firewall}
\author{Valentin Jonquière, Mathilde Chollon, Calliste Boudoux d'Hautefeuille}

\begin{document}

\maketitle
\pagebreak

\tableofcontents

\pagebreak

\section{Organisation du travail}
Nous travaillons sur le projet grâce à GitHub. Lors de chaque TD, nous créons des issues, que nous nous répartissons dans le Kanban.
Chaque membre du groupe possède un bout du TD à réaliser et aide les autres s'il finit en avance. Nous avons des branches pour chaque
 fonctionnalité et nous utilisons les merge requests afin de vérifier le travail des autres membres. Cela nous permet d'avoir du code
 fonctionnel dans la branche principale et de vraiment savoir ce que modifie chaque personne.


\section{Workflow ad-hoc}

\subsection{Snaphots}

\subsection{Sismos}
Au départ, nous ne possédions les données de sismos que d'un seul récepteur. De plus, ces données étaient "perdues", 
elles n'étaient pas sauvegardées, il était donc impossible de faire des traitements dessus.
Nous avons donc rajouté des fonctionnalités dans l'application afin de pouvoir gérer plusieurs récepteurs et sauvegarder les données dans un fichier.
Les options \textit{sismos} \textit{set-receivers} et \textit{sismos-folder} nous permettent respectivement d'activer la sauvegarde des sismos,
de choisir un fichier contentant les récepteurs à monitorer et le dossier de sauvegarde du fichier.
Lorsque l'on active l'option sismos, nous obtenons donc un fichier contenant sur chaque ligne le temps et la mesure de pression au récepteur.

Ce fichier peut ensuite être utilisé à des fins de visualisation, pour produire des courbes représentant la variabilité des pressions aux récepteurs données au cours du temps.

\section{Workflow in-situ}

\section{Comparaison workflows ad-hoc et in-situ}

\end{document}